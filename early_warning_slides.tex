\documentclass{beamer}
\usetheme{Boadilla}

% Override font size limits
\usepackage{anyfontsize}
% Select Computer Concrete as default font
\usepackage{concrete}
% Select Computer Concrete as math font
\usepackage{concmath}
\usepackage[T1]{fontenc}
% Set default font family as serif (sciposter sets sans serif by default)
\renewcommand{\familydefault}{\rmdefault}

\usepackage{xcolor}

\usepackage{amsmath,amssymb}
\input{macros}

\title[LIGO-Gxxxxxxx-vx]{Toward early-warning detection of compact binary coalescence}
%\subtitle[?]{?}
\institute[LIGO Caltech]{LIGO Laboratory, California Institute of Technology}
\author[L Singer]{Leo Singer}
\date{August 2, 2011}

\begin{document}


\frame{\titlepage}


\begin{frame}
\frametitle{Introduction}
\end{frame}


\section[Outline]{}
\frame{\tableofcontents}


\section[Motivation]{}


\begin{frame}
\frametitle{Motivation}
\end{frame}


\section[Prospects for early-warning detection]{}


\begin{frame}
	\frametitle{Detectability before merger}
	\begin{columns}
		\begin{column}{0.7\textwidth}
			\includegraphics[width=\textwidth]{figures/snr_psd}
		\end{column}
		\begin{column}{0.3\textwidth}
			In advanced \textsc{ligo}, inspiral signals are in principal detectable {\color{ink3}tens or hundreds of seconds} before the \textsc{gw} from the merger have reached the earth.
		\end{column}
	\end{columns}
\end{frame}

\begin{frame}
	\frametitle{Prospects for early-warning detection}
	\begin{center}
		\includegraphics[width=0.7\textwidth]{figures/snr_in_time}
	\end{center}

	Although rates are highly uncertain due to our small sample of confirmed binary neutron stars, of a predicted 40 detections~yr$^{-1}$,
	\begin{itemize}
		\item 10~yr$^{-1}$ will be detectable 10~s before merger, and
		\item 1~yr$^{-1}$ will be detectable 1~s before merger.
	\end{itemize}
\end{frame}

\begin{frame}
	\frametitle{Sky localization accuracy}
	Sky localization accuracy improves rapidly as the signal progresses toward merger.  Although a near real-time \textsc{gw} search pipeline might make it possible to point telescopes minutes after merger, one might hope to image counterparts to a few exceptional events before merger.

	\begin{center}
		\includegraphics[width=0.75\textwidth]{figures/loc_in_time}
	\end{center}
\end{frame}

\section[Method]{}

\begin{frame}
	\frametitle{Conventional inspiral searches: matched filter banks}
	General relativity predicts the \textsc{gw} signal due to the inspiral of a system with known intrinsic source parameters (mass, eccentricity, spin).
	\\~\\

	\begin{columns}
		\begin{column}{0.5\textwidth}
			\includegraphics[width=\textwidth]{figures/hexgrid}
			\begin{flushleft}
				\scriptsize{Image from Cokelar, T, Phys. Rev. D 76, 102004 (2007).}
			\end{flushleft}
		\end{column}
		\begin{column}{0.5\textwidth}
			To detect any signal that nature may provide, we can build banks of filters each of which has optimal signal to noise for a given source. \\~\\

			These matched filters tile the parameter space discretely, for example in a hexagonal grid. \\~\\~\\~\\
		\end{column}
	\end{columns}
\end{frame}

\begin{frame}
	\frametitle{Matched filter banks}
	For systems in which the effects of can be ignored, the intrinsic source parameters are just the component masses of the binary, $\theta = (m_1, m_2)$. \\~\\

	Strain observed by the detector is a linear combination of two orthogonal signals corresponding to the `+' and `$\times$' polarizations. \\~\\

	$M$ templates are chosen for $M/2$ sources $\theta_0$,~$\theta_1$,~$\dots$,~$\theta_{M/2-1}$. \\~\\

	For $i \in [0, M)$, unit normalized template $h_i [k]$, whitened detector data $x[k]$, the filter outputs are just cross-correlations:
	$$
		\rho_i [k] = \sum_{n=0}^{N-1} h_{i}[n] x [k-n].
	$$
\end{frame}

\begin{frame}
	\frametitle{Time domain method: \textsc{fir} filter}
	The most straightforward way to build a matched filter bank is using \textsc{fir} filters, which are just sliding dot products.

	\begin{columns}
		\begin{column}{0.6\textwidth}
			\includegraphics[width=\textwidth]{figures/fir}
			\begin{flushleft}
				\scriptsize{Image courtesy of Jonathan Blanchard, ``FIR Filter Canonical Realization,'' Wikipedia, February 23, 2008, \url{http://commons.wikimedia.org/wiki/File:FIR_Filter.svg}.}
			\end{flushleft}
		\end{column}
		\begin{column}{0.4\textwidth}
			Pros:
			\begin{itemize}
				\item Easy to implement
				\item Zero latency
			\end{itemize}
			Cons:
			\begin{itemize}
				\item Expensive if templates contain many samples
			\end{itemize}
		\end{column}
	\end{columns}
\end{frame}

\begin{frame}
	\frametitle{Frequency domain method: overlap-save}
	\begin{columns}
		\begin{column}{0.5\textwidth}
			An alternative to the time domain method is frequency domain convolution via the \textsc{fft}.

			Pros:
			\begin{itemize}
				\item Computationally efficient even for very long templates
				\item Highly tuned \textsc{fft}s available for most architectures
			\end{itemize}
			Cons:
			\begin{itemize}
				\item Input must be zero-padded, output must be clipped
				\item High latency: typically comparable to length of templates
			\end{itemize}
		\end{column}
		\begin{column}{0.5\textwidth}
			\includegraphics[width=0.9\textwidth]{figures/overlap-save}
			\begin{flushright}
				\scriptsize{Image courtesy of Douglas Jones, ``Fast Convolution,'' Connexions, June 21, 2004, \url{http://cnx.org/content/m12022/1.5/}.}
			\end{flushright}
		\end{column}
	\end{columns}
\end{frame}

\begin{frame}
	\frametitle{Frequency domain method: overlap-save}
	The frequency domain (\textsc{fd}) method's latency is determined by the overlap between blocks. \\~\\
	\begin{columns}
		\begin{column}{0.6\textwidth}
			\includegraphics[width=\textwidth]{figures/fd_latency}
		\end{column}
		\begin{column}{0.4\textwidth}

			The latency can be made arbitrary small, but at the price of a {\color{ink3}\emph{divergent}} computational cost of
			\begin{equation*}
				\approx 2 \left(1 + \frac{\textrm{filter length}}{\textrm{latency}}\right)
			\end{equation*}
			operations~/~sample~/ $\lg (\textrm{template length})$.
		\end{column}
	\end{columns}
\end{frame}

\begin{frame}
\frametitle{Method}
\end{frame}


\section[Implementation]{}


\begin{frame}
\frametitle{Implementation}
\end{frame}


\section[Results]{}


\begin{frame}
\frametitle{Results}
\end{frame}


\section[Conclusion]{}


\begin{frame}
\frametitle{Conclusion}
\end{frame}



\end{document}
