\documentclass[portrait,plainboxedsections]{sciposter}

\usepackage{amsmath}
\usepackage{amssymb}
\usepackage{booktabs}
\usepackage{graphicx}
\usepackage{microtype}
\usepackage{concmath}
\usepackage{aas_macros}
\usepackage{xcolor}

\input{macros}

% customize colors
\definecolor{BoxCol}{RGB}{0,35,102} % RoyalBlue
\definecolor{SectionCol}{RGB}{255,255,255} % White


\title{Toward early-warning detection of compact binary coalescence}

\author{\textsc{Nickolas V. Fotopoulos}, Leo P. Singer}

\institute{LIGO Laboratory, California Institute of Technology}

\email{foton@caltech.edu}

\begin{document}
\conference{Amaldi 9, Cardiff, Wales, UK; This document bears the DCC number LIGO-G1100683-v1.}

\maketitle

\begin{minipage}[t]{0.25\textwidth}

\section*{Motivation}

\PARstart{A}s a compact binary system loses energy to gravitational waves (\GW{}s), its
orbital separation decays, leading to a runaway inspiral with the \GW{}
amplitude and frequency increasing until the system eventually merges.  If a
neutron star (\textsc{ns}) is involved, it may become tidally disrupted near
the merger and fuel an electromagnetic (\EM{}) counterpart~%
\cite{shibata:2007}.  Effort from both the \GW{} and astronomy communities may make it
possible to use \GW{} observations as an early warning trigger for \EM{}
followup.
%
\begin{figure}[h]
\includegraphics[width=\textwidth]{figures/snr_in_time}
\caption{\label{fig:earlywarning}Expected number of \textsc{ns}--\textsc{ns}
sources that could be detectable by Advanced \LIGO\ a given number of seconds
before coalescence.  The heavy solid line is the most realistic yearly rate
estimate.  The shaded region represents the 5 to 95\% confidence interval
arising from uncertainty in predicted event rates~\cite{Abadie:2010p10836}.}
\end{figure}
%
\begin{figure}[h]
\includegraphics[width=\textwidth]{figures/loc_in_time}
\caption{\label{fig:sky-localization-accuracy}Area of the 90\% confidence
region as a function of time before coalescence for sources with anticipated
detectability rates of 40, 10, 1, and 0.1~yr$^{-1}$. The heavy dot indicates
the time at which the accumulated \SNR\ exceeds a threshold of~8.}
\end{figure}

There were a number of sources of latency associated with the search for
\CBC{} signals in S6/VSR3~\cite{HugheyGWPAW2011}:

\begin{itemize}
\item Data acquisition and aggregation ($\gtrsim$100~ms)

\item Data conditioning ($\sim$1~min)

\item Trigger generation (2--5~min)

\item Alert generation (2--3~min)

\item Human validation (10--20~min)

\end{itemize}

\end{minipage}%
\hspace{0.05\textwidth}%
\begin{minipage}[t]{0.4\textwidth}

\begin{figure}[h!]
	\begin{center}
		\includegraphics[width=\textwidth]{figures/lloid-diagram}
		\caption{\label{fig:pipeline} Schematic of \lloid{} algorithm illustrating
signal flow.  Circles with arrows represent interpolation
\protect\includegraphics[scale=3]{figures/upsample-symbol} or decimation
\protect\includegraphics[scale=3]{figures/downsample-symbol}.  Circles with plus
signs represent summing junctions
\protect\includegraphics[scale=3]{figures/adder-symbol}.  Squares
\protect\includegraphics[scale=3]{figures/fir-symbol} stand for \fir{} filters.  Sample
rate decreases from the top of the diagram to the bottom.  In this diagram each
time slice contains three \fir\ filters that are linearly combined to produce
four output channels.  In a typical pipeline the number of \fir\ filters is
much less than the number of output channels.}
	\end{center}
\end{figure}

\begin{equation*}
	\rho_i^s [k] =%
		% Interpolation SNR
		\color{\diagramcoloraccum}
		\overbrace{
			\left(H^\uparrow \rho_i^{s+1}\right)[k]
		}^\text{\clap{{\sc snr} from prev. time slices}}
		% Plus ...
		\;\;+\;\;
		% Reconstruction
		\color{\diagramcolorreconstruct}
		\underbrace{
			\sum_{l=0}^{L^s-1} v_{il}^s \sigma_l^s
		}_\text{\clap{reconstruction}}
		% Orthogonal FIR filter
		\color{\diagramcolorfir}
		\;\overbrace{
			\sum_{n=0}^{N^s-1} u_l^s[n] \color{black}x^s[k-n]
		}^\text{\clap{orthogonal {\sc fir} filters}} .
\end{equation*}

\section*{Results}

\begin{table}
\caption{\label{table:flops}Computational cost in \flops\ of the \TD\ method, the \FD\ method, and \lloid\ for the sub-bank described in Figure~\ref{fig:tmpltbank}.}
\begin{center}
\begin{tabular}{lll}
\toprule
method & \flops\ & latency (s) \\
\midrule
time domain & $2.4\times10^{13}$ & $0$ \\
frequency domain & $2.6\times10^8$ & $2\times10^3$ \\
\lloid\ (theory) & $4.7\times10^8$ & $2\times10^{-3}$ \\
\lloid\ (prototype) & ----------- & $5$ \\
\bottomrule
\end{tabular}
\end{center}
\end{table}

\PARstart{W}e have demonstrated a computationally feasible filtering algorithm
for the rapid and even early-warning detection of \GW{}s emitted during the
coalescence of neutron stars and stellar-mass black holes.  It is one part
of a complicated analysis and observation strategy that will unfortunately
have other sources of latency. However, we hope that it will motivate
further work to reduce such technical sources of \GW{} observation latency
and encourage the possibility of even more rapid \EM\ followup observations
to catch prompt emission in the advanced detector era.

\end{minipage}%
\hspace{0.05\textwidth}%
\begin{minipage}[t]{0.25\textwidth}

\section*{Implementation}

\PARstart{A}ccuracy and latency are critical parameters to verify. We present
measurements based on a \gstreamer{}-based prototype of the \lloid{}
algorithm. Figures \ref{fig:tmpltbank} and \ref{fig:time_slices} and
Table~\ref{tab:time_slices} depict the masses used for the test and the
resulting time-slice design.

\begin{figure}[h]
	\includegraphics[width=\textwidth]{figures/tmpltbank}
	\caption{\label{fig:tmpltbank}Source parameters selected for sub-bank used in this
case study, consisting of component masses $m_1$, $m_2$, between 1 and 3~$M_\odot$, and
chirp masses $\mathcal{M}$ between 1.1955 and 1.2045~$M_\odot$.}
\end{figure}

\begin{figure}
\includegraphics[width=\textwidth]{figures/envelope}
\caption{\label{fig:time_slices} Time slice parameters overlaid onto a representative waveform amplitude envelope.}
\end{figure}

\begin{table}
\centering
\begin{tabular}{rr@{,\,}lcc}
\toprule
\\ [-2ex]
$f^s$ & $[t^s$&$t^{s+1})$ & & \\% [1ex]
\\[-2.5ex]
(Hz) & \multicolumn{2}{c}{(s)} & $N^s$ & $L^s$ \\ \midrule
4096 & [0&0.5) & 2048 & 8 \\
512 & [0.5&4.5) & 2048 & 10 \\
256 & [4.5&12.5) & 2048 & 10 \\
128 & [12.5&76.5) & 8192 & 28 \\
64 & [76.5&140.5) & 4096 & 18 \\
64 & [140.5&268.5) & 8192 & 25 \\
64 & [268.5&396.5) & 8192 & 20 \\
32 & [396.5&460.5) & 2048 & 9 \\
32 & [460.5&588.5) & 4096 & 16 \\
32 & [588.5&844.5) & 8192 & 26 \\
32 & [844.5&1100.5) & 8192 & 12 \\
\bottomrule
\end{tabular}
\caption{\label{tab:time_slices} Filter design sub-bank of 657 templates.  From left to right, this table shows the sample rate, time interval, number of samples, and number of orthogonal templates for each time slice.  The \SVD{} tolerance is set to 0.9999.}
\end{table}

\end{minipage}

\section*{Acknowledgements}

This poster was prepared by \textsc{nf} and \textsc{ls} based on a preprint
article designated LIGO-P0900004 and authored by Kipp Cannon, Romain Cariou,
Mireia Crispin-Ortuzar, \textsc{nf}, Melissa Frei, Chad Hann, Erin Kara, Drew
Keppel, Laura Liao, Stephen Privitera, \textsc{ls}, and Alan Weinstein. \LIGO\
was constructed by the California Institute of Technology and Massachusetts
Institute of Technology with funding from the National Science Foundation and
operates under cooperative agreement \textsc{phy}-\oldstylenums{0107417}. This
research is supported by the National Science Foundation through a Graduate
Research Fellowship to \textsc{ls}.

%%% References

\bibliographystyle{plain}
\bibliography{references}

\end{document}
